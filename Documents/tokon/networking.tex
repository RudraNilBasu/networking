\documentclass[12pt]{article}

%\usepackage{fullpage}
%\usepackage{epic}
%\usepackage{eepic}
%\usepackage{graphicx}

\usepackage{listings} % Code
\usepackage{fancyhdr} % header footer
\usepackage{xcolor}  % Color
\usepackage{mathtools} % Math
\usepackage{tabularx}
\usepackage{csvsimple, filecontents}

\usepackage{geometry}
 \geometry{
 a4paper,
 total={170mm,257mm},
 left=20mm,
 top=20mm,
 }

%\newcommand{\proof}[1]{
%{\noindent {\it Proof.} {#1} \rule{2mm}{2mm} \vskip \belowdisplayskip}
%}


%\newtheorem{lemma}{Lemma}[section]
%\newtheorem{theorem}[lemma]{Theorem}
%\newtheorem{claim}[lemma]{Claim}
%\newtheorem{definition}[lemma]{Definition}
%\newtheorem{corollary}[lemma]{Corollary}

%\setlength{\oddsidemargin}{0in}
%\setlength{\topmargin}{0in}
%\setlength{\textwidth}{6.5in}
%\setlength{\textheight}{8.5in}

\cfoot{footer}

\lstset {
%language=Java,
backgroundcolor = \color{white},
                   language = C++,
                   xleftmargin = 2mm,
                   framexleftmargin = 1em
%lineskip={-1.5pt}
}

%\usepackage[utf8]{inputenc}
 
 
% Information about contents section
%\title{Contents}
%\author{Rudra Nil Basu}
\date{ }
 
%\renewcommand*\contentsname{Summary}

\begin{document}

% to generate the contents page
%\maketitle
\tableofcontents

\newpage

%\setlength{\fboxrule}{.5mm}\setlength{\fboxsep}{1.2mm}
%\newlength{\boxlength}\setlength{\boxlength}{\textwidth}
%\addtolength{\boxlength}{-4mm}
%\begin{center}\framebox{\parbox{\boxlength}{\bf
CS692: Network Lab \hfill 
Year: 2017
%Date: 11/10/2016
%\\
%DATE
%\hfill
%}}\end{center}
\vspace{5mm}

\section{Chat Server using TCP}

\subsection{Server}

\textit{Code.}

\begin{lstlisting}
#include <stdio.h>
#include <sys/types.h> 
#include <sys/socket.h>
#include <netinet/in.h>
#include <stdlib.h>
#include <strings.h>
#include <unistd.h>

void error(char *msg)
{
	perror(msg);
	exit(1);
}

int main(int argc, char *argv[])
{
	int sockfd, newsockfd, portno, clilen;
	char buffer[256];
	struct sockaddr_in serv_addr, cli_addr;
	int n;
	if (argc < 2) {
		fprintf(stderr,"ERROR, no port provided\n");
		exit(1);
	}
	sockfd = socket(AF_INET, SOCK_STREAM, 0);
	if (sockfd < 0) 
		error("ERROR opening socket");
	bzero((char *) &serv_addr, sizeof(serv_addr));
	portno = atoi(argv[1]);
	serv_addr.sin_family = AF_INET;
	serv_addr.sin_addr.s_addr = INADDR_ANY;
	serv_addr.sin_port = htons(portno);
	if (bind(sockfd, (struct sockaddr *) &serv_addr,
				sizeof(serv_addr)) < 0) 
		error("ERROR on binding");
	listen(sockfd,5);
	clilen = sizeof(cli_addr);
	newsockfd = accept(sockfd, (struct sockaddr *) &cli_addr, 
	&clilen);
	if (newsockfd < 0) 
		error("ERROR on accept");
	bzero(buffer,256);
	n = read(newsockfd,buffer,255);
	if (n < 0) error("ERROR reading from socket");
	printf("Here is the message: %s\n",buffer);
	n = write(newsockfd,"I got your message",18);
	if (n < 0) error("ERROR writing to socket");
	return 0; 
}

\end{lstlisting}

\subsection{Client}

\begin{lstlisting}
#include <stdio.h>
#include <sys/types.h>
#include <sys/socket.h>
#include <netinet/in.h>
#include <netdb.h> 
#include <stdlib.h>
#include <unistd.h>
#include <string.h>

void error(char *msg)
{
	perror(msg);
	exit(0);
}

int main(int argc, char *argv[])
{
	int sockfd, portno, n;

	struct sockaddr_in serv_addr;
	struct hostent *server;

	char buffer[256];
	if (argc < 3) {
		fprintf(stderr,"usage %s hostname port\n", argv[0]);
		exit(0);
	}
	portno = atoi(argv[2]);
	sockfd = socket(AF_INET, SOCK_STREAM, 0);
	if (sockfd < 0) 
		error("ERROR opening socket");
	server = gethostbyname(argv[1]);
	if (server == NULL) {
		fprintf(stderr,"ERROR, no such host\n");
		exit(0);
	}
	bzero((char *) &serv_addr, sizeof(serv_addr));
	serv_addr.sin_family = AF_INET;
	bcopy((char *)server->h_addr, 
			(char *)&serv_addr.sin_addr.s_addr,
			server->h_length);
	serv_addr.sin_port = htons(portno);
	if (connect(sockfd,(struct sockaddr *)&serv_addr,
	sizeof(serv_addr)) < 0) 
		error("ERROR connecting");
	printf("Please enter the message: ");
	bzero(buffer,256);
	fgets(buffer,255,stdin);
	n = write(sockfd,buffer,strlen(buffer));
	if (n < 0) 
		error("ERROR writing to socket");
	bzero(buffer,256);
	n = read(sockfd,buffer,255);
	if (n < 0) 
		error("ERROR reading from socket");
	printf("%s\n",buffer);
	return 0;
}
\end{lstlisting}

\subsection{Manual}

\begin{lstlisting}
gcc server.c -o _server
./_server 8000

gcc client.c -o _client
./_client 127.0.0.1 8000
\end{lstlisting}

\subsection{Output}

\begin{lstlisting}
# From server side
Here is the message: Hello World

# From client side
Enter your message: Hello World
I got your message
\end{lstlisting}

\section{Chat Server using UDP}

\subsection{Server}

\textit{Code.}

\begin{lstlisting}
#include <stdio.h>
#include <netinet/in.h>
#include <sys/types.h>
#include <sys/socket.h>
#include <netdb.h>
#include <string.h>
#include <stdlib.h>
#include <unistd.h> /* For close()*/

#define MAX 80
#define PORT 43454
#define SA struct sockaddr

/*
 * listen function: To listen from client
 */
void listen_client(int sockfd)
{
	char buff[MAX];
	int n,clen;
	struct sockaddr_in cli;
	clen = sizeof(cli);
	for(;;) {
		bzero(buff,MAX);
		recvfrom(sockfd,buff,sizeof(buff),0,(SA *)&cli,&clen);
		printf("From client %s To client\n",buff);
		//bzero(buff,MAX);
		n = 0;
		//while ((buff[n++]=getchar()) != '\n');
		sendto(sockfd,buff,sizeof(buff),0,(SA *)&cli,clen);
		if(strncmp("exit",buff,4) == 0)
		{
			printf("Server Exit...\n");
			break;
		}
	}
}
int main()
{
	int sockfd;
	struct sockaddr_in servaddr;
	sockfd=socket(AF_INET,SOCK_DGRAM,0);
	if(sockfd == -1) {
		printf("socket creation failed...\n");
		exit(1);
	}
	else {
		printf("Socket successfully created..\n");
	}
	bzero(&servaddr,sizeof(servaddr));
	servaddr.sin_family = AF_INET;
	servaddr.sin_addr.s_addr = htonl(INADDR_ANY);
	servaddr.sin_port = htons(PORT);
	if ((bind(sockfd,(SA *)&servaddr,sizeof(servaddr))) != 0) {
		printf("socket bind failed...\n");
		exit(1);
	}
	else {
		printf("Socket successfully binded..\n");
	}
	listen_client(sockfd);
	close(sockfd);
}

\end{lstlisting}

\subsection{Client}

\begin{lstlisting}
#include <sys/socket.h>
#include <netdb.h>
#include <string.h>
#include <stdlib.h>
#include <stdio.h>
#include <unistd.h>
#include <arpa/inet.h>
#include <time.h>

#define MAX 80
#define PORT 43454
#define SA struct sockaddr

int main()
{
	char buff[MAX];
	int sockfd,len,n;
	struct sockaddr_in servaddr;
	sockfd = socket(AF_INET,SOCK_DGRAM,0);
	if(sockfd == -1) {
		printf("socket creation failed...\n");
		exit(1);
	}
	else {
		printf("Socket successfully created..\n");
	}
	bzero(&servaddr,sizeof(len));
	servaddr.sin_family = AF_INET;
	servaddr.sin_addr.s_addr = inet_addr("127.0.0.1");
	servaddr.sin_port = htons(PORT);
	len = sizeof(servaddr);
	for(;;) {
		printf("\nEnter string : ");
		n = 0;
		while ((buff[n++]=getchar()) != '\n');
		sendto(sockfd,buff,sizeof(buff),0,(SA *)&servaddr,len);
		bzero(buff,sizeof(buff));
		recvfrom(sockfd,buff,sizeof(buff),0,(SA *)&servaddr,&len);
		printf("From Server : %s\n",buff);
		time_t current_time = time(NULL);
		printf("%s\n",ctime(&current_time));
		if(strncmp("exit",buff,4) == 0) {
			printf("Client Exit...\n");
			break;
		}
	}
	close(sockfd);
}
\end{lstlisting}

\subsection{Manual}

\begin{lstlisting}
gcc server.c -o server
./server

gcc client.c -o client
./client
\end{lstlisting}

%\subsection{Output}

%\begin{lstlisting}
%# From server side

%# From client side
%\end{lstlisting}

\section{Dat time server using TCP}
\subsection{server}

\begin{lstlisting}
#include <sys/socket.h>
#include <sys/types.h>
#include <netinet/in.h>
#include <netdb.h>
#include <stdio.h>
#include <time.h>
#include <stdlib.h>
#include <netinet/in.h>
#include <arpa/inet.h>
#include <string.h>
#include <unistd.h>

int main(int argc, char **argv)
{
    int listenfd, connfd;
	int port = atoi(argv[1]);
 
    struct sockaddr_in servaddr;
    char buff[1000];
    time_t ticks;
    listenfd = socket(AF_INET, SOCK_STREAM, 0);

    bzero(&servaddr, sizeof(servaddr));
    servaddr.sin_family = AF_INET;
    servaddr.sin_addr.s_addr = htonl(INADDR_ANY);
    servaddr.sin_port = htons(port);

    bind(listenfd, (struct sockaddr *) &servaddr, sizeof(servaddr));

    listen(listenfd, 8);
    for (;;) {
		connfd = accept(listenfd, (struct sockaddr *) NULL, NULL);

		ticks = time(NULL);
		snprintf(buff, sizeof(buff), "%.24s\r\n", ctime(&ticks));
		write(connfd, buff, strlen(buff));
		close(connfd);
    }
}
\end{lstlisting}

\subsection{client}

\begin{lstlisting}
#include <stdio.h>
#include <sys/types.h>
#include <sys/socket.h>
#include <netinet/in.h>
#include <netdb.h> 
#include <stdlib.h>
#include <unistd.h>
#include <string.h>

void error(char *msg)
{
	perror(msg);
	exit(1);
}

int main(int argc, char *argv[])
{
	int sockfd, portno, n;

	struct sockaddr_in serv_addr;
	struct hostent *server;

	char buffer[256];
	if (argc < 3) {
		fprintf(stderr,"usage %s hostname port\n", argv[0]);
		exit(0);
	}
	portno = atoi(argv[2]);
	sockfd = socket(AF_INET, SOCK_STREAM, 0);
	if (sockfd < 0) 
		error("ERROR opening socket");
	server = gethostbyname(argv[1]);
	if (server == NULL) {
		fprintf(stderr,"ERROR, no such host\n");
		exit(0);
	}
	bzero((char *) &serv_addr, sizeof(serv_addr));
	serv_addr.sin_family = AF_INET;
	bcopy((char *)server->h_addr, 
			(char *)&serv_addr.sin_addr.s_addr,
			server->h_length);
	serv_addr.sin_port = htons(portno);
	if (connect(sockfd,(struct sockaddr *)&serv_addr,
	sizeof(serv_addr)) < 0) 
		error("ERROR connecting");
	while (1) {
		printf("Please enter the message: ");
		bzero(buffer,256);
		fgets(buffer,255,stdin);
		n = write(sockfd,buffer,strlen(buffer));
		if (n < 0) 
			error("ERROR writing to socket");
		bzero(buffer,256);
		n = read(sockfd,buffer,255);
		if (n < 0) 
			error("ERROR reading from socket");
		printf("%s\n",buffer);
	}
	return 0;
}
\end{lstlisting}

\subsection{Manual}

\begin{lstlisting}
gcc server.c -o server
./server 8000

gcc client.c -o client
./client 127.0.0.1 8000
\end{lstlisting}

%\subsection{Output}

%\begin{lstlisting}
%# From server side

%# From client side
%\end{lstlisting}

\section{Math Server}
\subsection{server}

\begin{lstlisting}
#include <stdio.h>
#include <sys/types.h> 
#include <sys/socket.h>
#include <netinet/in.h>
#include <stdlib.h>
#include <string.h>
#include <unistd.h>

void error(char *msg)
{
	perror(msg);
	exit(1);
}

int string_to_int(char *);
float calculate(int,int,char);

int main(int argc, char *argv[])
{
	int sockfd, newsockfd, portno, clilen;
	char buffer[256];
	struct sockaddr_in serv_addr, cli_addr;
	int n;
	if (argc < 2) {
		fprintf(stderr,"ERROR, no port provided\n");
		exit(1);
	}
	sockfd = socket(AF_INET, SOCK_STREAM, 0);
	if (sockfd < 0) 
		error("ERROR opening socket");
	bzero((char *) &serv_addr, sizeof(serv_addr));
	portno = atoi(argv[1]);
	serv_addr.sin_family = AF_INET;
	serv_addr.sin_addr.s_addr = INADDR_ANY;
	serv_addr.sin_port = htons(portno);
	if (bind(sockfd, (struct sockaddr *) &serv_addr,
				sizeof(serv_addr)) < 0) 
		error("ERROR on binding");
	printf("-------------------------\n");
	printf("Running server\n");
	listen(sockfd,5);
	clilen = sizeof(cli_addr);
	printf("accepting\n");
	newsockfd = accept(sockfd, (struct sockaddr *) &cli_addr, &clilen);
	printf("accepted\n");
	if (newsockfd < 0) 
		error("ERROR on accept");
	int num, first_num, second_num;
	float result;
	char operation;
	for(int i = 0; i < 3; i++) {
		bzero(buffer,256);
		n = read(newsockfd,buffer,255);
		if (n < 0) error("ERROR reading from socket");
		if (i == 0 || i == 2) {
			num = string_to_int(buffer);
			printf("We got: %d\n", num);
			if (i == 0) {
				first_num = num;
			} else {
				second_num = num;
			}
		} else {
			operation = buffer[0];
			printf("Operation: %c\n",operation);
		}
		if (i != 2) {
			n = write(newsockfd, "Recieved", 8);
		} else {
			result = calculate(first_num, second_num, operation);
			char msg[] = "Result: ";
			char final_msg[100];
			sprintf(final_msg, "%s%f", msg, result);
			n = write(newsockfd, final_msg, sizeof(final_msg));
		}
		if (n < 0) error("ERROR writing to socket");
	}

	return 0; 
}

int string_to_int(char *str)
{
	int num = 0, len = strlen(str), i;
	for(i = 0;i < len;i++) {
		if(str[i] == '\n') {
			break;
		}
		num = (num * 10) + (str[i] - '0');
	}
	return num;
}

float calculate(int a, int b, char op)
{
	if (op == '+') {
		return a+b;
	} else if (op == '-') {
		return a-b;
	} else if (op == '*') {
		return a*b;
	} else if (op == '/') {
		return ((float)(a * 1.0)/b);
	}
}

\end{lstlisting}

\subsection{client}

\begin{lstlisting}
#include <stdio.h>
#include <sys/types.h>
#include <sys/socket.h>
#include <netinet/in.h>
#include <netdb.h> 
#include <stdlib.h>
#include <unistd.h>
#include <string.h>

void error(char *msg)
{
	perror(msg);
	exit(0);
}

int main(int argc, char *argv[])
{
	int sockfd, portno, n;

	struct sockaddr_in serv_addr;
	struct hostent *server;

	char buffer[256];
	if (argc < 3) {
		fprintf(stderr,"usage %s hostname port\n", argv[0]);
		exit(0);
	}
	portno = atoi(argv[2]);
	sockfd = socket(AF_INET, SOCK_STREAM, 0);
	if (sockfd < 0) 
		error("ERROR opening socket");
	server = gethostbyname(argv[1]);
	if (server == NULL) {
		fprintf(stderr,"ERROR, no such host\n");
		exit(0);
	}
	bzero((char *) &serv_addr, sizeof(serv_addr));
	serv_addr.sin_family = AF_INET;
	bcopy((char *)server->h_addr, 
			(char *)&serv_addr.sin_addr.s_addr,
			server->h_length);
	serv_addr.sin_port = htons(portno);
	if (connect(sockfd,(struct sockaddr *)&serv_addr,sizeof(serv_addr)) < 0) 
		error("ERROR connecting");
	for (int i = 0;i < 3; i++) {
		printf("Please enter the message: ");
		bzero(buffer,256);
		fgets(buffer,255,stdin);
		n = write(sockfd,buffer,strlen(buffer));
		if (n < 0) 
			error("ERROR writing to socket");
		bzero(buffer,256);
		n = read(sockfd,buffer,255);
		if (n < 0) 
			error("ERROR reading from socket");
		printf("%s\n",buffer);
	}
	return 0;
}
\end{lstlisting}

\subsection{Manual}

\begin{lstlisting}
gcc server.c -o server
./server 8000

gcc client.c -o client
./client 127.0.0.1 8000
\end{lstlisting}

%\subsection{Output}

%\begin{lstlisting}
%# From server side

%# From client side
%\end{lstlisting}

\section{Concurrent Server}
\subsection{server}

\begin{lstlisting}
#include <stdio.h>
#include <stdlib.h>
#include <netinet/in.h>
#include <arpa/inet.h>
#include <sys/types.h>
#include <sys/socket.h>
#include <string.h>
#include <unistd.h>

int main()
{
	int sockfd, newsockfd;
	int clilen;
	struct sockaddr_in cli_addr, serv_addr;
	int i;
	char buff[100];
	if ((sockfd = socket(AF_INET, SOCK_STREAM, 0)) < 0) {
		printf("Cannot create socket\n");
		exit(1);
	}
	serv_addr.sin_family = AF_INET;
	serv_addr.sin_addr.s_addr = INADDR_ANY;
	serv_addr.sin_port = htons(6000); // port: 6000
	if (bind(sockfd, (struct sockaddr *) &serv_addr, 
	sizeof(serv_addr)) < 0) {
		printf("Unable to bind local address\n");
		exit(1);
	}
	listen(sockfd, 5); // upto 5 concurrent clients
	while (1) {
		clilen = sizeof(cli_addr);
		newsockfd = accept(sockfd, (struct sockaddr *) &cli_addr,
		 &clilen);
		printf("hola\n");
		if (newsockfd < 0) {
			printf("Accept error\n");
			exit(1);
		}
		if (fork() == 0) {
			close(sockfd);
			while (1) {
				strcpy(buff, "Message from server");
				send(newsockfd, buff, strlen(buff) + 1, 0);
				for (i = 0;i < 100;i++) {
					buff[i] = '\0';
				}
				recv(newsockfd, buff, 100, 0);
				printf("%s\n",buff);
			}
			close(newsockfd);
			exit(0);
		}
		close(newsockfd);
	}
}
\end{lstlisting}

\subsection{client}

\begin{lstlisting}
#include <stdio.h>
#include <sys/types.h>
#include <sys/socket.h>
#include <netinet/in.h>
#include <netdb.h> 
#include <stdlib.h>
#include <unistd.h>
#include <string.h>

void error(char *msg)
{
	perror(msg);
	exit(1);
}

int main(int argc, char *argv[])
{
	int sockfd, portno, n;

	struct sockaddr_in serv_addr;
	struct hostent *server;

	char buffer[256];
	if (argc < 3) {
		fprintf(stderr,"usage %s hostname port\n", argv[0]);
		exit(0);
	}
	portno = atoi(argv[2]);
	sockfd = socket(AF_INET, SOCK_STREAM, 0);
	if (sockfd < 0) 
		error("ERROR opening socket");
	server = gethostbyname(argv[1]);
	if (server == NULL) {
		fprintf(stderr,"ERROR, no such host\n");
		exit(0);
	}
	bzero((char *) &serv_addr, sizeof(serv_addr));
	serv_addr.sin_family = AF_INET;
	bcopy((char *)server->h_addr, 
			(char *)&serv_addr.sin_addr.s_addr,
			server->h_length);
	serv_addr.sin_port = htons(portno);
	if (connect(sockfd,(struct sockaddr *)&serv_addr,sizeof(serv_addr)) < 0) 
		error("ERROR connecting");
	while (1) {
		printf("Please enter the message: ");
		bzero(buffer,256);
		fgets(buffer,255,stdin);
		n = write(sockfd,buffer,strlen(buffer));
		if (n < 0) 
			error("ERROR writing to socket");
		bzero(buffer,256);
		n = read(sockfd,buffer,255);
		if (n < 0) 
			error("ERROR reading from socket");
		printf("%s\n",buffer);
	}
	return 0;
}
\end{lstlisting}

\subsection{Manual}

\begin{lstlisting}
gcc server.c -o server
./server 6000

gcc client.c -o client
./client 127.0.0.1 6000
\end{lstlisting}

%\subsection{Output}

%\begin{lstlisting}
%# From server side

%# From client side
%\end{lstlisting}

\section{File Transfer Protocol}
\subsection{server}

\begin{lstlisting}
#include<stdio.h> 
#include<sys/types.h> 
#include<netinet/in.h> 
#include<string.h> 
#include<stdlib.h>

int main(int argc,char *argv[])
{
	FILE *fp,*fp2;
	int sockfd,newsockfd,portno,clilen,n,i;
	size_t  max = 100;
	char fname[100],name[100],fname1[100],arg[100],arg1[100];
	struct sockaddr_in serv_addr,cli_addr;
	if (argc < 2) 
	{
		fprintf(stderr,"ERROR, no port provided\n");
		exit(1);
	}
	sockfd = socket(AF_INET, SOCK_STREAM, 0);
	if (sockfd < 0) 
		error("ERROR opening socket");
	bzero((char *) &serv_addr, sizeof(serv_addr));
	portno = atoi(argv[1]);
	serv_addr.sin_family = AF_INET;
	serv_addr.sin_addr.s_addr = INADDR_ANY;
	serv_addr.sin_port = htons(portno);
	if (bind(sockfd, (struct sockaddr *) &serv_addr,
	sizeof(serv_addr)) < 0) 
		error("ERROR on binding");
	listen(sockfd,5);
	clilen = sizeof(cli_addr);
	newsockfd = accept(sockfd, (struct sockaddr *) &cli_addr,
	 &clilen);
	if(newsockfd<0)
		printf("error on accept\n");
	memset(fname1,'\0',100);
	memset(arg,'\0',100);
	memset(arg1,'\0',100);
	n=recv(newsockfd,fname,100,0);
	fname[n]='\0';
	strcpy(fname1,"find . -name ");
	strcat(fname1,fname);
	printf("%s\n",fname1);
	system(fname1);
	strcat(fname1," >> 11.txt");
	printf("%s\n",fname1);
	system(fname1);
	system("cat 11.txt");
	fp2=fopen("11.txt","r");
	fgets(arg,100,fp2);
	arg[strlen(arg)-1]='\0';
	printf("%s\n",arg);
	if(n<0)
		printf("error on read");
	else
	{
		fp=fopen(arg,"r"); //read mode
		if(fp==NULL)
		{
			send(newsockfd,"error",5,0);
			close(newsockfd);
		}
		else
		{
			while(fgets(name,100,fp))
			{
				if(write(newsockfd,name,100)<0)
				{
					printf("can't send\n");
				}
			}
			if(!fgets(name,sizeof(name),fp)) 
	  		{	 
	   			send(newsockfd,"Done",4,0);
	  		} 
			return 0;
		}
	}
}
\end{lstlisting}

\subsection{client}

\begin{lstlisting}
#include<stdio.h>
#include<stdlib.h> 
#include<sys/socket.h> 
#include<netinet/in.h> 
#include<stdlib.h>
#include<string.h>
int main(int argc,char *argv[]) 
{ 
	FILE *fp;
	int sockfd,newsockfd,portno,r;
	char fname[100],fname1[100],text[100];
	struct sockaddr_in serv_addr;
	portno = atoi(argv[2]);
	sockfd=socket(AF_INET,SOCK_STREAM,0);
	if(sockfd<0)
	{
		printf("Error on socket creation\n");
		exit(0);
	}
	else
		printf("socket created\n");
	serv_addr.sin_family=AF_INET; 
	serv_addr.sin_addr.s_addr=inet_addr(argv[1]);
	serv_addr.sin_port=htons(portno); 
	if(connect(sockfd,(struct sockaddr*)&serv_addr,
	sizeof(serv_addr))<0) 
	{
		printf("Error in Connection...\n"); 
		exit(0);
	}
	else 
		printf("Connected...\n"); 
	printf("Enter the filename existing in the server:\n");
	scanf("%s",fname);
	printf("Enter the filename to be written to:\n");
	scanf("%s",fname1);
	fp=fopen(fname1,"w");
	send(sockfd,fname,100,0);
	while(1)
	{
		r=recv(sockfd,text,100,0);
		text[r]='\0';
		fprintf(fp,"%s",text);
		if(strcmp(text,"error")==0)
			printf("file not available\n");
		if(strcmp(text,"Done")==0)
		{
			printf("file is transferred\n");
			fclose(fp);
			close(sockfd);
			break;
		}
		else
			fputs(text,stdout);
	}
	return 0;
}
\end{lstlisting}

\subsection{Manual}

\begin{lstlisting}
gcc server.c -o server
./server

gcc client.c -o client
./client
\end{lstlisting}

%\subsection{Output}

%\begin{lstlisting}
%# From server side

%# From client side
%\end{lstlisting}

\section{Multicast Server}
\subsection{server}

\begin{lstlisting}
#include <sys/types.h>
#include <sys/socket.h>
#include <netinet/in.h>
#include <arpa/inet.h>
#include <time.h>
#include <string.h>
#include <stdio.h>


#define HELLO_PORT 12345
#define HELLO_GROUP "225.0.0.37"

main(int argc, char *argv[])
{
    struct sockaddr_in addr;
    int fd, cnt;
    struct ip_mreq mreq;
    char *message="Hello, World!";

			     /* create what looks like an ordinary UDP socket */
    if ((fd=socket(AF_INET,SOCK_DGRAM,0)) < 0) {
	perror("socket");
	exit(1);
    }

    /* set up destination address */
    memset(&addr,0,sizeof(addr));
    addr.sin_family=AF_INET;
    addr.sin_addr.s_addr=inet_addr(HELLO_GROUP);
    addr.sin_port=htons(HELLO_PORT);

    /* now just sendto() our destination! */
    while (1) {
	if (sendto(fd,message,sizeof(message),0,(struct sockaddr *) &addr,
		    sizeof(addr)) < 0) {
	    perror("sendto");
	    exit(1);
	}
	sleep(1);
    }
}
\end{lstlisting}

\subsection{client}

\begin{lstlisting}
#include <sys/types.h>
#include <sys/socket.h>
#include <netinet/in.h>
#include <arpa/inet.h>
#include <time.h>
#include <string.h>
#include <stdio.h>


#define HELLO_PORT 12345
#define HELLO_GROUP "225.0.0.37"
#define MSGBUFSIZE 256

main(int argc, char *argv[])
{
    struct sockaddr_in addr;
    int fd, nbytes,addrlen;
    struct ip_mreq mreq;
    char msgbuf[MSGBUFSIZE];

    u_int yes=1;            /*** MODIFICATION TO ORIGINAL */

    /* create what looks like an ordinary UDP socket */
    if ((fd=socket(AF_INET,SOCK_DGRAM,0)) < 0) {
	perror("socket");
	exit(1);
					           }


    /**** MODIFICATION TO ORIGINAL */
    /* allow multiple sockets to use the same PORT number */
    if (setsockopt(fd,SOL_SOCKET,SO_REUSEADDR,&yes,sizeof(yes)) < 0) {
	perror("Reusing ADDR failed");
	exit(1);
    }
    /*** END OF MODIFICATION TO ORIGINAL */

    /* set up destination address */
    memset(&addr,0,sizeof(addr));
    addr.sin_family=AF_INET;
    addr.sin_addr.s_addr=htonl(INADDR_ANY); /* N.B.: differs from sender */
    addr.sin_port=htons(HELLO_PORT);

    /* bind to receive address */
    if (bind(fd,(struct sockaddr *) &addr,sizeof(addr)) < 0) {
	perror("bind");
	exit(1);
    }

    /* use setsockopt() to request that the kernel join a multicast group */
    mreq.imr_multiaddr.s_addr=inet_addr(HELLO_GROUP);
    mreq.imr_interface.s_addr=htonl(INADDR_ANY);
    if (setsockopt(fd,IPPROTO_IP,IP_ADD_MEMBERSHIP,&mreq,sizeof(mreq)) < 0) {
	perror("setsockopt");
	exit(1);
    }

    /* now just enter a read-print loop */
    while (1) {
	addrlen=sizeof(addr);
	if ((nbytes=recvfrom(fd,msgbuf,MSGBUFSIZE,0,
			(struct sockaddr *) &addr,&addrlen)) < 0) {
	    perror("recvfrom");
	    exit(1);
	}
	puts(msgbuf);
    }
}
\end{lstlisting}

\subsection{Manual}

\begin{lstlisting}
gcc server.c -o server
./server

gcc client.c -o client
./client
\end{lstlisting}

\subsection{Output}

\begin{lstlisting}
# From server side

# From client side
\end{lstlisting}

\section{Broadcast Server}
\subsection{server}

\begin{lstlisting}
#include <stdio.h>      /* for printf() and fprintf() */
#include <sys/socket.h> /* for socket(), connect(), sendto(), and recvfrom() */
#include <arpa/inet.h>  /* for sockaddr_in and inet_addr() */
#include <stdlib.h>     /* for atoi() and exit() */
#include <string.h>     /* for memset() */
#include <unistd.h>     /* for close() */

#define MAXRECVSTRING 255  /* Longest string to receive */

void DieWithError(char *errorMessage);  /* External error handling function */

int main(int argc, char *argv[])
{
    int sock;                         /* Socket */
    struct sockaddr_in broadcastAddr; /* Broadcast Address */
    unsigned short broadcastPort;     /* Port */
    char recvString[MAXRECVSTRING+1]; /* Buffer for received string */
    int recvStringLen;                /* Length of received string */

    if (argc != 2)    /* Test for correct number of arguments */
    {
        fprintf(stderr,"Usage: %s <Broadcast Port>\n", argv[0]);
        exit(1);
    }

    broadcastPort = atoi(argv[1]);   /* First arg: broadcast port */

    /* Create a best-effort datagram socket using UDP */
    if ((sock = socket(PF_INET, SOCK_DGRAM, IPPROTO_UDP)) < 0)
        perror("socket() failed");

    /* Construct bind structure */
    memset(&broadcastAddr, 0, sizeof(broadcastAddr));   /* Zero out structure */
    broadcastAddr.sin_family = AF_INET;                 /* Internet address family */
    broadcastAddr.sin_addr.s_addr = htonl(INADDR_ANY);  /* Any incoming interface */
    broadcastAddr.sin_port = htons(broadcastPort);      /* Broadcast port */

    /* Bind to the broadcast port */
    if (bind(sock, (struct sockaddr *) &broadcastAddr, sizeof(broadcastAddr)) < 0)
        perror("bind() failed");

    /* Receive a single datagram from the server */
    if ((recvStringLen = recvfrom(sock, recvString, MAXRECVSTRING, 0, NULL, 0)) < 0)
        perror("recvfrom() failed");

    recvString[recvStringLen] = '\0';
    printf("Received: %s\n", recvString);    /* Print the received string */
    
    close(sock);
    exit(0);
}
\end{lstlisting}

\subsection{client}

\begin{lstlisting}
#include <stdio.h>      /* for printf() and fprintf() */
#include <sys/socket.h> /* for socket() and bind() */
#include <arpa/inet.h>  /* for sockaddr_in */
#include <stdlib.h>     /* for atoi() and exit() */
#include <string.h>     /* for memset() */
#include <unistd.h>     /* for close() */

void DieWithError(char *errorMessage);  /* External error handling function */

int main(int argc, char *argv[])
{
    int sock;                         /* Socket */
    struct sockaddr_in broadcastAddr; /* Broadcast address */
    char *broadcastIP;                /* IP broadcast address */
    unsigned short broadcastPort;     /* Server port */
    char *sendString;                 /* String to broadcast */
    int broadcastPermission;          /* Socket opt to set permission to broadcast */
    unsigned int sendStringLen;       /* Length of string to broadcast */

    if (argc < 4)                     /* Test for correct number of parameters */
    {
        fprintf(stderr,"Usage:  %s <IP Address> <Port> <Send String>\n", argv[0]);
        exit(1);
    }

    broadcastIP = argv[1];            /* First arg:  broadcast IP address */ 
    broadcastPort = atoi(argv[2]);    /* Second arg:  broadcast port */
    sendString = argv[3];             /* Third arg:  string to broadcast */

    /* Create socket for sending/receiving datagrams */
    if ((sock = socket(PF_INET, SOCK_DGRAM, IPPROTO_UDP)) < 0)
        perror("socket() failed");

    /* Set socket to allow broadcast */
    broadcastPermission = 1;
    if (setsockopt(sock, SOL_SOCKET, SO_BROADCAST, (void *) &broadcastPermission, 
          sizeof(broadcastPermission)) < 0)
        perror("setsockopt() failed");

    /* Construct local address structure */
    memset(&broadcastAddr, 0, sizeof(broadcastAddr));   /* Zero out structure */
    broadcastAddr.sin_family = AF_INET;                 /* Internet address family */
    broadcastAddr.sin_addr.s_addr = inet_addr(broadcastIP);/* Broadcast IP address */
    broadcastAddr.sin_port = htons(broadcastPort);         /* Broadcast port */

    sendStringLen = strlen(sendString);  /* Find length of sendString */
    for (;;) /* Run forever */
    {
         /* Broadcast sendString in datagram to clients every 3 seconds*/
         if (sendto(sock, sendString, sendStringLen, 0, (struct sockaddr *) 
               &broadcastAddr, sizeof(broadcastAddr)) != sendStringLen)
             perror("sendto() sent a different number of bytes than expected");

        sleep(3);   /* Avoids flooding the network */
    }
    /* NOT REACHED */
}
\end{lstlisting}

\subsection{Manual}

\begin{lstlisting}
gcc server.c -o server
./server

gcc client.c -o client
./client
\end{lstlisting}

\subsection{Output}

\begin{lstlisting}
# From server side

# From client side
\end{lstlisting}

%======================= Template=======================
%\section{}
%\subsection{server}

%\begin{lstlisting}
%\end{lstlisting}

%\subsection{client}

%\begin{lstlisting}
%\end{lstlisting}

%\subsection{Manual}

%\begin{lstlisting}
%gcc server.c -o server
%./server

%gcc client.c -o client
%./client
%\end{lstlisting}

%\subsection{Output}

%\begin{lstlisting}
%# From server side

%# From client side
%\end{lstlisting}
%======================= Template=======================
\end{document}
%\grid\grid
\grid
